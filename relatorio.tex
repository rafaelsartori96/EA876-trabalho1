% Rafael Sartori M. dos Santos, 186154
% Letícia Mayumi A. Tateishi, 201454
\documentclass[brazilian,a4paper,twocolumn]{article}

% Título
\title{EA876 -- Trabalho 1}
\author{Letícia Mayumi A. Tateishi, 201454 \\Rafael Sartori M. Santos, 186154}
\date{26 de abril de 2019}

% Configuração do documento
\setlength{\parskip}{3pt}
\usepackage[utf8]{inputenc} % tipo de documento UTF-8
\usepackage{mathtools} % permitir expressões matemáticas
\usepackage{babel} % configuração da lingua portuguesa
\usepackage{caption} % para legenda de tabelas e figuras
\usepackage{cleveref} % para referenciar tabelas e figuras melhor
\usepackage{indentfirst} % indentação de todo primeiro parágrafo

\newcommand{\s}{\;\;}

% Início do documento
\begin{document}

\maketitle

\section{Introdução}

    Neste trabalho, desenvolvemos um compilador de expressões matemáticas para linguagem de montagem de ARM. Consideramos, nas expressões matemáticas, apenas números inteiros e parênteses, com as operações de soma, subtração e multiplicação. A partir da entrada, utilizamos Flex e Bison para imprimir no terminal um código em assembly de ARM que retorna em R0 o resultado dos cálculos matemáticos.
    
\section{Método}
    
    Inicialmente, determinamos os tokens necessários para o código Flex: \texttt{INT, SOMA, SUBTRACAO, MULTIPLICACAO, EOL, ABRE\_PAR} e \texttt{FECHA\_PAR}. A partir desses tokens, priorizamos a operação de multiplicação em relação à soma e à subtração e desenvolvemos a seguinte gramática livre de contexto:
    
    $P \xrightarrow{} E \s \texttt{EOL}$
    
    $E \xrightarrow{} \texttt{INT}$
    
    $E \xrightarrow{} \texttt{ABRE\_PAR} \s E \s \texttt{FECHA\_PAR}$
    
    $E \xrightarrow{} E \s \texttt{MULTIPLICACAO} \s E$
    
    $E \xrightarrow{} E \s \texttt{SOMA} \s E$
    
    $E \xrightarrow{} E \s \texttt{SUBTRACAO} \s E$
    

    
\section{Resultados}


\end{document}
