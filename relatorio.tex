% Rafael Sartori M. dos Santos, 186154
% Letícia Mayumi A. Tateishi, 201454
\documentclass[brazilian,a4paper,twocolumn]{article}

% Título
\title{EA876 -- Trabalho 1}
\author{Letícia Mayumi A. Tateishi, 201454\\Rafael Sartori M. Santos, 186154}
\date{26 de abril de 2019}

% Configuração do documento
\setlength{\parskip}{3pt}
\usepackage[utf8]{inputenc} % tipo de documento UTF-8
\usepackage{mathtools} % permitir expressões matemáticas
\usepackage{babel} % configuração da lingua portuguesa
\usepackage{caption} % para legenda de tabelas e figuras
\usepackage{cleveref} % para referenciar tabelas e figuras melhor
\usepackage{indentfirst} % indentação de todo primeiro parágrafo

\newcommand{\s}{\;\;}

% Início do documento
\begin{document}

\maketitle

\section{Introdução}

    Neste trabalho, desenvolvemos um compilador de expressões matemáticas para linguagem de montagem de ARM. Consideramos, nas expressões matemáticas, apenas números inteiros e parênteses, com as operações de soma, subtração e multiplicação. A partir da entrada, utilizamos Flex e Bison para imprimir no terminal um código assembly de ARM que termina com o resultado dos cálculos matemáticos no registrador \texttt{r0}.

\section{Método}

    Inicialmente, determinamos os tokens necessários para o código Flex: \texttt{INT ($-?[0-9]+$), SOMA ($+$), SUBTRACAO ($-$), MULTIPLICACAO ($*$), EOL (\textbackslash{}n), ABRE\_PAR (()} e \texttt{FECHA\_PAR ())}. A partir desses tokens, priorizamos a operação de multiplicação em relação à soma e à subtração e desenvolvemos a seguinte gramática livre de contexto:

    $P \xrightarrow{} E \s \texttt{EOL}$

    $E \xrightarrow{} \texttt{INT}$

    $E \xrightarrow{} \texttt{ABRE\_PAR} \s E \s \texttt{FECHA\_PAR}$

    $E \xrightarrow{} E \s \texttt{MULTIPLICACAO} \s E$

    $E \xrightarrow{} E \s \texttt{SOMA} \s E$

    $E \xrightarrow{} E \s \texttt{SUBTRACAO} \s E$

    Na gramática, $P$ representa o programa e $E$ representa uma expressão.

    A ideia principal que possibilita o desenvolvimento do código assembly de ARM é o uso da pilha para armazenar os inteiros iniciais que foram encontrados e os resultados parciais das operações assim que foram calculados, pois não haveria registradores suficientes para vários parêntesis aninhados.

    Para isso, o código ARM produzido para o caso em que a expressão é um inteiro deve empilhar o registrador \texttt{r0} que deve possuir seu valor. Já para casos de expressões com operações (seja soma, subtração ou multiplicação), devemos desempilhar dois inteiros para os registradores \texttt{r0} e \texttt{r1}, fazer a operação, armazená-la em \texttt{r0} e empilhar o resultado. Dessa maneira, garantimos que os resultados parciais e o resultado final estarão sempre em \texttt{r0}.

\section{Resultados}

    Como saída, o programa imprime o código em linguagem de montagem de ARM que, quando compilado e executado, termina com o resultado da expressão matemática dada como entrada em \texttt{r0}.

    Por exemplo, a expressão $ 2 * (1 + (-3)) $ produz o seguinte código, que foi comentado posteriormente para entendimento humano:

\begin{verbatim}
    ldr r0, =2          ; carrega 2
    str r0, [sp, #4]!   ; empilha 2
    ldr r0, =1          ; carrega 1
    str r0, [sp, #4]!   ; empilha 1
    ldr r0, =-3         ; carrega -3
    str r0, [sp, #4]!   ; empilha -3
    ldr r0, [sp], #-4   ; desempilha -3
    ldr r1, [sp], #-4   ; desempilha 1
    add r0, r0, r1      ; soma 1 com -3
    str r0, [sp, #4]!   ; empilha -2
    ldr r0, [sp], #-4   ; desempilha -2
    ldr r1, [sp], #-4   ; desempilha 2
    bl multiplicar      ; multiplica -2*2
    str r0, [sp, #4]!   ; empilha -4
    end                 ; r0 mantém -4
    ; função multiplicar anexa ao fim
\end{verbatim}

\end{document}
